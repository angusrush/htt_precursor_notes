\documentclass[main.tex]{subfiles}

\begin{document}

\chapter{Model categories}
\label{ch:model_categories}

\section{Localization of categories}
\label{sec:localization_of_categories}

Many areas of mathematics have a notion of \emph{weak equivalence,} i.e.\ a class of morphisms which are not isomorphisms but which one wants to think of as equivalences.
\begin{itemize}
  \item In topology, one often wants to study spaces up to homotopy equivalence.

  \item In homological algebra, one is often interested in chain complexes only up to quasi-isomorphism.

  \item In ordinary category theory, one wants to consider two categories to be the same if they are equivalent.
\end{itemize}

The difficulty in treating these classes of morphisms as equivalences is that we would like equivalences to be invertible, whereas morphisms belonging to the above classes are in general not invertible. Therefore, we would like to solve the following problem: given a category $\mathcal{C}$ and a set of morphisms $\W$ in $\mathcal{C}$, how can we change $\mathcal{C}$ so as to make each morphism in $\W$ an isomorphism? There are two ways that one could go about this.
\begin{enumerate}
  \item One could freely add formal inverses for all morphisms in $\W$. This is, modulo issues of size, always possible, but difficult to work with. This is analogous to the localization of a ring.

  \item One can try to find a replacement for each object of $\mathcal{C}$ such that weak equivalences between these objects turn out to be invertible. This is what one does in homological algebra by taking projective resolutions.
\end{enumerate}

Whatever approach one takes, the result is called a the localization of $\mathcal{C}$ by $\W$.

\begin{definition}[localization of a category]
  \label{def:localization_of_a_category}
  Let $\mathcal{C}$ be a category, and $\mathcal{W}$ a set of morphisms of $\mathcal{C}$. The \defn{localization of $\mathcal{C}$ at $\mathcal{W}$} is a pair $(\mathcal{C}[\mathcal{W}^{-1}], \pi)$ where $\mathcal{C}([\mathcal{W}^{-1}])$ is a category and $\pi\colon \mathcal{C} \to \mathcal{C}[\mathcal{W}^{-1}]$ is a functor, such that the following conditions hold.

  \begin{itemize}
    \item For all $w \in \mathcal{W}$, $\pi(w)$ is an isomorphism.

    \item For any category $\mathcal{A}$ and functor $F\colon \mathcal{C} \to \mathcal{A}$ such that $F(w)$ is an isomorphism for all $w \in \mathcal{W}$, there exists a functor $\tilde{F}\colon \mathcal{C}[\mathcal{W}^{-1}] \to \mathcal{A}$ and a natural isomorphism $\tilde{F} \circ \pi \Rightarrow F$.
      \begin{equation*}
        \begin{tikzcd}[column sep=small]
          \mathcal{C}
          \arrow[rr, swap, ""'{name=A}, swap, "F"]
          \arrow[dr, swap, "\pi"]
          && \mathcal{A}
          \\
          &\mathcal{C}[\mathcal{W}^{-1}]
          \arrow[ur, swap, "\exists\tilde{F}"]
          \arrow[from=A, Leftarrow]
        \end{tikzcd}
      \end{equation*}
      Note that the above diagram is not required to commute on the nose.

    \item For every category $\mathcal{A}$, the pullback functor
      \begin{equation*}
        \pi^{*}\colon \mathbf{Cat}(\mathcal{C}[\mathcal{W}^{-1}], \mathcal{A}) \to \mathbf{Cat}(\mathcal{C}, \mathcal{A})
      \end{equation*}
      is fully faithful.
  \end{itemize}
\end{definition}

Note that by the third axiom, the functor $\tilde{F}$ is guaranteed to be unique up to isomorphism: fully faithfulness tells us precisely that functors $\mathcal{C} \to \mathcal{A}$ and functors $\mathcal{C}[\mathcal{W}^{-1}] \to \mathcal{A}$ sending $\mathcal{W}$ to isomorphisms are in bijection up to isomorphism

%The idea behind the localization of a category is very similar to that of a ring. Roughly speaking, in localizing a category $\mathcal{C}$ at a set of morphisms $\mathcal{W}$, one adds just enough morphisms to $\mathcal{C}$ so that all the morphisms in $\mathcal{W}$ have inverses. Of course in adding any morphism to a category, one also has to freely add in compositions with other morphisms. For this reason, the localization of a category is in general unwieldy. Furthermore, one runs into size-theoretic issues: even if $\mathcal{C}$ is a small category, one cannot in general be sure that $\mathcal{C}[\mathcal{W}^{-1}]$ is small.

As previously alluded to, there is an explicit construction of the localization of a category at any set of weak equivalences, but it is difficult to work with. Fortunately, model categories provide a way of computing and working with localizations which is practical and avoids issues of size.

\section{Model categories}
\label{sec:model_categories}

Model categories provide a distillation of the structures appearing in various flavors of homotopy theory.

\begin{definition}[model category]
  \label{def:model_category}
  A \defn{model category} is a category $\mathcal{C}$ together with three sets of morphisms
  \begin{itemize}
    \item $\W$, the set of \emph{weak equivalences}

    \item $\fib$, the set of \emph{fibrations}

    \item $\cof$, the set of \emph{cofibrations}
  \end{itemize}
  subject to the following axioms.
  \begin{enumerate}[label=(M\arabic*)]
    \item\label{item:MC_complete_cocomplete} The category $\mathcal{C}$ has all small limits and colimits.

    \item\label{item:MC_two_out_of_three} The set $\W$ satisfies the so-called \emph{two-out-of-three law:} if the diagram
      \begin{equation*}
        \begin{tikzcd}
          X
          \arrow[rr, "g"]
          \arrow[rd, swap, "h"]
          && Z
          \\
          & Y
          \arrow[ur, swap, "f"]
        \end{tikzcd}
      \end{equation*}
      commutes and two out of three of $f$, $g$, and $h$ are weak equivalences, then the third is as well.

    \item\label{item:MC_closed_under_retracts} The sets $\W$, $\fib$, and $\cof$ are all closed under retracts: given the following diagram, if $g$ belongs to any of the above sets, then so does $f$.
      \begin{equation*}
        \begin{tikzcd}
          X
          \arrow[r]
          \arrow[d, swap, "f"]
          & Y
          \arrow[r]
          \arrow[d, "g"]
          & X
          \arrow[d, "f"]
          \\
          Z
          \arrow[r]
          & W
          \arrow[r]
          & Z
        \end{tikzcd}
      \end{equation*}

    \item\label{item:MC_solve_some_lifting_problems} We can solve a lifting problem
      \begin{equation*}
        \begin{tikzcd}
          U
          \arrow[r]
          \arrow[d, swap, "i"]
          & X
          \arrow[d, "p"]
          \\
          V
          \arrow[r]
          & P
        \end{tikzcd}
      \end{equation*}
      if all of the following conditions hold.
      \begin{itemize}
        \item The morphism $i$ is a cofibration.

        \item The morphism $p$ is a fibration.

        \item At least one of $i$ and $p$ is a weak equivalence.
      \end{itemize}

    \item\label{item:MC_morphisms_factorize} Any morphism $f\colon X \to Y$ admits the following factorizations.
      \begin{itemize}
        \item We can write $f = p \circ i$, where $p$ is a fibration and $i$ is both a cofibration and a weak equivalence.
          \begin{equation*}
            \begin{tikzcd}
              A
              \arrow[rr, "f"]
              \arrow[dr, swap, "\cof \cap \W \ni i"]
              && B
              \\
              & X
              \arrow[ur, swap, "p \in \fib"]
            \end{tikzcd}
          \end{equation*}

        \item We can write $f = p \circ i$, where $p$ is both a fibration and a weak equivalence, and $i$ is a cofibration.
          \begin{equation*}
            \begin{tikzcd}
              A
              \arrow[rr, "f"]
              \arrow[dr, swap, "\cof \ni i"]
              && B
              \\
              & X
              \arrow[ur, swap, "p \in \fib \cap \W"]
            \end{tikzcd}
          \end{equation*}
      \end{itemize}
  \end{enumerate}
\end{definition}

To simplify terminology, we will use the following terms.
\begin{itemize}
  \item We will call morphisms which are both weak equivalences and fibrations \emph{trivial fibrations}:
    \begin{equation*}
      \W \cap \fib = \{\text{trivial fibrations}\}.
    \end{equation*}

  \item We will call morphisms which are both weak equivalences and cofibrations \emph{trivial cofibrations}:
    \begin{equation*}
      \W \cap \cof = \{\text{trivial cofibrations}\}.
    \end{equation*}

  \item We will call an object $X \in \mathcal{C}$ \emph{cofibrant} if the unique morphism $\emptyset \to X$ is a cofibration.

  \item We will call an object $Y \in \mathcal{C}$ \emph{fibrant} if the unique morphism $X \to *$ is a fibration.
\end{itemize}

We will see some examples of model categories after a few lemmas.

\begin{lemma}[lifting lemma]
  \label{lemma:retract_argument}
  Let $\mathcal{C}$ be any category, and let $f = p \circ i$ be a morphism in $\mathcal{C}$.
  \begin{enumerate}
    \item If $f$ has the left lifting property with respect to $p$, then $f$ is a retract of $i$.

    \item If $f$ has the right lifting property with respect to $i$ then $f$ is a retract of $p$.
  \end{enumerate}
\end{lemma}
\begin{proof}
  Suppose $f$ has the left-lifting property with respect to $p$. Then the following diagram admits a lift.
  \begin{equation*}
    \begin{tikzcd}
      A
      \arrow[r, "i"]
      \arrow[d, swap, "f"]
      & X
      \arrow[d, "p"]
      \\
      B
      \arrow[r, equals]
      \arrow[ur, dashed]
      & B
    \end{tikzcd}
  \end{equation*}

  Expanding, we find the following diagram.
  \begin{equation*}
    \begin{tikzcd}
      A
      \arrow[r, equals]
      \arrow[d, swap, "f"]
      & A
      \arrow[d, "i"]
      \arrow[r, equals]
      & A
      \arrow[d, "f"]
      \\
      B
      \arrow[r, dashed]
      \arrow[rr, bend right, swap, "\id_{B}"]
      & X
      \arrow[r, "p"]
      & B
    \end{tikzcd}
  \end{equation*}
  This exhibits $f$ as a retract of $i$.

  The second case is dual.
\end{proof}

The axioms for a model category are over-specified, in the sense that providing $\W$ and either $\fib$ and $\cof$ is sufficient to derive the other one.
\begin{lemma}
  \label{lemma:relations_between_lifting_properties_and_fibrations}
  Let $\mathcal{C}$ be a model category. We have the following.
  \begin{enumerate}
    \item $\cof\rlp = \fib \cap \W$
    \item $\cof = \llp(\fib \cap \W)$
    \item $(\cof \cap \W)\rlp = \fib$
    \item $\cof \cap \W = \llp\fib$
  \end{enumerate}
\end{lemma}
\begin{proof}
  We prove 4. The proof of 1 is the same, mutatis mutandis.

  Let $f \in \cof \cap \W$. Then by \hyperref[item:MC_solve_some_lifting_problems]{Axiom~\ref*{item:MC_solve_some_lifting_problems}}, $f \in \llp\fib$.

  Now suppose $f \in \llp\fib$. By \hyperref[item:MC_morphisms_factorize]{Axiom~\ref*{item:MC_morphisms_factorize}}, we can always find a factorization $f = p \circ i$, where $p$ is a fibration and $i$ is a trivial cofibration.

  By \hyperref[item:MC_solve_some_lifting_problems]{Axiom~\ref*{item:MC_solve_some_lifting_problems}}, $f$ has the left lifting property with respect to $p$, so by \hyperref[lemma:retract_argument]{Lemma~\ref*{lemma:retract_argument}}, $f$ is a retract of $i$. But since by \hyperref[item:MC_closed_under_retracts]{Axiom~\ref*{item:MC_closed_under_retracts}} both $\W$ and $\cof$ are closed under retracts, $f$ is also a trivial cofibration as needed.

  From 4, 3 follows immediately. The proof of 1 is precisely the same as that of 4, and 2 follows from 1 immediately.
\end{proof}

\begin{corollary}
  \label{cor:trivial_fibrations_and_cofibrations_are_saturated}
  In any model category, the sets $\cof$ and $\cof \cap \W$ are saturated (\hyperref[def:saturated]{Definition~\ref*{def:saturated}}).
\end{corollary}
\begin{proof}
  By \hyperref[thm:left_perp_is_saturated]{Theorem~\ref*{thm:left_perp_is_saturated}}, any set which is of the form $\llp \mathcal{M}$ is saturated.
\end{proof}

\begin{example}
  \label{eg:model_categories}
  The following are examples of model structures.
  \begin{enumerate}
    \item The category $\mathbf{Top}$ carries the so-called \emph{Quillen model structure,} with
      \begin{itemize}
        \item $\W =$ weak homotopy equivalences,

        \item $\fib =$ Serre fibrations,

        \item $\cof =$ $\llp(\fib \cap \W)$.
      \end{itemize}

    \item The category $\SSet$ has a model structure called the \emph{Kan model structure} with
      \begin{itemize}
        \item $\W =$ weak homotopy equivalences, i.e.\ Kan fibrations $X \to Y$ such that $\pi_{0}(X) \to \pi_{0}(Y)$ is a bijection and $\pi_{n}(X, x) \to \pi_{n}(Y, f(x))$ is an isomorphism for all $x \in X$.

        \item $\fib =$ Kan fibrations,

        \item $\cof =$ monomorphisms.
      \end{itemize}

      Note that in this case, $\W \cap \cof = \llp\fib = \{\text{anodyne morphisms}\}$.

    \item For any small ring $R$, the category $\mathbf{Ch}_{\geq}(R)$ of bounded-below chain complexes of small left $R$-modules has a model structure, called the \emph{projective model structure,} with
      \begin{itemize}
        \item $\W = $ Quasi-isomorphisms

        \item $\fib = $ morphisms $f$ of complexes such that $f_{n}$ is an epimorphism for all $n \geq 0$

        \item $\cof = $ morphisms $f$ of complexes such that $f_{n}$ is a monomorphism for all $n \geq 0$, and $\coker f_{n}$ is projective for all $n \geq 0$.
      \end{itemize}

    %\item For any small ring $R$, the category $\mathbf{Ch}^{\leq}(R)$ of bounded-below cochain complexes of small left $R$-modules has a model structure, called the \emph{injective model structure,} with
    %  \begin{itemize}
    %    \item $\W = $ Quasi-isomorphisms

    %    \item $\fib = $ morphisms $f$ of complexes such that $f_{n}$ is an epimorphism for all $n \geq 0$, and $\ker f_{n}$ injective for all $n \geq 0$.

    %    \item $\cof = $ morphisms $f$ of complexes such that $f_{n}$ is a monomorphism for all $n \geq 0$.
    %  \end{itemize}
  \end{enumerate}
\end{example}

\begin{example}
  \label{eg:opposite_model_category}
  Let $\mathcal{C}$ be a category with model structure $(\W, \fib, \cof)$. Then $\mathcal{C}\op$ is a model category with model structure $(\W, \cof, \fib)$.
\end{example}

\subsection{A model structure on the category of categories}
\label{ssc:a_model_structure_on_the_category_of_categories}

Proving that three sets of morphisms form the fibrations, cofibrations, and weak equivalences of a model structure can be an immense undertaking. There are a few model structures, however, which are simple enough that the conditions can be checked relatively simply.

Consider the category $I$ with the following objects and morphisms (excluding identity morphisms).
\begin{equation*}
  \begin{tikzcd}
    0
    \arrow[r, bend left]
    & 1
    \arrow[l, bend left]
  \end{tikzcd}
\end{equation*}

Denote the functor including $0 \hookrightarrow I$ by $\iota$.

The following is a model structure on $\mathbf{Cat}$.
\begin{itemize}
  \item $\W =$ categorical equivalences

  \item $\fib =$ \emph{isofibrations}, i.e.\ functors with the right lifting property with respect to $\iota$

  \item $\cof =$ functors which are injective on objects
\end{itemize}

We verify the axioms one by one. This section is incomplete and a bit wrong at the moment. I'll come back to it.
\subsubsection{Axiom M1: Limits and colimits}

We already know that $\mathbf{Cat}$ has all limits and colimits.

\subsubsection{Axiom M2: Two-out-of-three}

We have the following results.
\begin{itemize}
  \item If a functor is a weak equivalence, then its weak inverse is also a weak equivalence.

  \item Composition of weak equivalences gives again a weak equivalence (by the horizontal composition formula).
\end{itemize}
Thus, the same reasoning that shows that the 2/3 law holds for bona-fide isomorphisms also holds for weak equivalences.

\subsubsection{Axiom M3: Closure under retracts}
First suppose that $G$ is an equivalence of categories. We need to show that $F$ is also an equivalence of categories.
\begin{equation*}
  \begin{tikzcd}
    \mathcal{A}
    \arrow[d, swap, "F"]
    \arrow[r, "R"]
    & \mathcal{C}
    \arrow[d, "G"]
    \arrow[r, "S"]
    & \mathcal{A}
    \arrow[d, "F"]
    \\
    \mathcal{B}
    \arrow[r, swap, "T"]
    & \mathcal{D}
    \arrow[r, swap, "U"]
    & \mathcal{B}
  \end{tikzcd}
\end{equation*}
Since both $S \circ R$ and $U \circ T$ compose to the identity, $T$ and $R$ are injective on objects and morphisms, and $S$ and $U$ are similarly surjective. Thus $G$ is:
\begin{itemize}
  \item \textbf{Essentially surjective:} We can map any $b \in \mathcal{B}$ to $\mathcal{D}$ with $T$, then lift it to something in $\mathcal{C}$ using $G$'s weak inverse, which we'll call $G ^{-1}$. Mapping the result of this to $\mathcal{A}$ with $S$ gives us an object
    \begin{equation*}
      (S \circ G^{-1} \circ T)(b) \in A.
    \end{equation*}

    The claim is that under $F$ this maps to something isomorphic to $b$. Evaluating on $F$, we have
    \begin{equation*}
      F((S \circ G^{-1} \circ T)(b)) = (U \circ G \circ G^{-1} \circ T)(b).
    \end{equation*}
    We have a natural isomorphism $G \circ G^{-1} \Rightarrow \id$, which tells us that
    \begin{equation*}
      (U \circ G \circ G^{-1} \circ T)(b) \cong (U \circ T)(b) = b.
    \end{equation*}

  \item \textbf{Fully faithful:} The logic is very similar. We can map any morphism $f$ in $\mathcal{B}$ to a morphism in $\mathcal{D}$ with $T$, and (because $G$ is fully faithful) find a unique preimage in $\mathcal{C}$ which we'll call $G^{-1}(T(f))$. Applying $S$ gives us a morphism
    \begin{equation*}
      (S \circ G^{-1} \circ T)(f) \quad\text{in}\ \mathcal{A}.
    \end{equation*}
    Then
    \begin{align*}
      (F \circ S \circ G^{-1} \circ T)(f) &= (U \circ G \circ G^{-1} \circ T)(f) \\
      &= (U \circ T)(f) \\
      &= f.
    \end{align*}
\end{itemize}

Now suppose that $G$ is in $\mathfrak{C}$. We need to show that $F \in \mathfrak{C}$.

Applying the forgetful functor to $\mathbf{Set}$ we have the following diagram.
\begin{equation*}
  \begin{tikzcd}
    A
    \arrow[r, "r"]
    \arrow[d, swap, "f"]
    & C
    \arrow[d, "g"]
    \arrow[r, "s"]
    & A
    \arrow[d, "f"]
    \\
    B
    \arrow[r, swap, "t"]
    & D
    \arrow[r, swap, "u"]
    & B
  \end{tikzcd}
\end{equation*}
The map $r$ is injective because the identity factors through it, and $g$ is injective by assumption. Suppose we are given $a$, $a' \in A$ such that $f(a) = f(a')$. Then $t(f(a)) = t(f(a'))$, so $g(r(a)) = g(r(a'))$. But $g$ and $r$ are both injective, so $a = a'$.

Now suppose that $G$ is in $\mathfrak{F}$. We need to show that $F \in \mathfrak{F}$.
\begin{equation*}
  \begin{tikzcd}
    \mathcal{A}
    \arrow[d, swap, "F"]
    \arrow[r, "R"]
    & \mathcal{C}
    \arrow[d, "G"]
    \arrow[r, "S"]
    & \mathcal{A}
    \arrow[d, "F"]
    \\
    \mathcal{B}
    \arrow[r, swap, "T"]
    & \mathcal{D}
    \arrow[r, swap, "U"]
    & \mathcal{B}
  \end{tikzcd}
\end{equation*}

To see this, note that we can augment any lifting problem
\begin{equation*}
  \begin{tikzcd}
    (*)
    \arrow[r]
    \arrow[d, hookrightarrow]
    & \mathcal{A}
    \arrow[d, "F"]
    \\
    (* \leftrightarrow \bullet)
    \arrow[r]
    & \mathcal{B}
  \end{tikzcd}
\end{equation*}
as follows; composing with $R$ gives us a lifting problem we can solve, yielding the dashed arrow.
\begin{equation*}
  \begin{tikzcd}
    (*)
    \arrow[r]
    \arrow[d, hookrightarrow]
    & \mathcal{A}
    \arrow[d]
    \arrow[r, "R"]
    & \mathcal{C}
    \arrow[r, "S"]
    \arrow[d]
    & \mathcal{A}
    \arrow[d]
    \\
    (* \leftrightarrow \bullet)
    \arrow[r]
    \arrow[urr, dashed]
    & \mathcal{B}
    \arrow[r]
    & \mathcal{D}
    \arrow[r]
    & \mathcal{B}
  \end{tikzcd}
\end{equation*}
We can compose this with $S$ to get a lift to $A$, which is what we want.

\subsubsection{Axiom M4: Solving lifting problems}
Consider a diagram
\begin{equation*}
  \begin{tikzcd}
    \mathcal{A}
    \arrow[r, "F"]
    \arrow[d, swap, "i"]
    & \mathcal{B}
    \arrow[d, "P"]
    \\
    \mathcal{C}
    \arrow[r, swap, "G"]
    & \mathcal{D}
  \end{tikzcd}
\end{equation*}
where $i$ is injective on objects and $P$ is both an isofibration and a weak equivalence.

Pick any $d \in \mathcal{D}$. Since $P$ is an equivalence of categories, there is some object $b \in \mathcal{B}$ such that $P(b) \cong d$. Define a functor $I \to \mathcal{D}$ which sends $* \mapsto b$ and $\bullet \mapsto P(b)$, and a functor $(*) \to \mathcal{B}$ which picks out $b$. This gives us a commuting diagram
\begin{equation*}
  \begin{tikzcd}
    (*)
    \arrow[r]
    \arrow[d]
    & \mathcal{B}
    \arrow[d]
    \\
    I
    \arrow[r]
    & \mathcal{D}
  \end{tikzcd}
\end{equation*}
Since $P$ is an isofibration, we have a lift $I \to \mathcal{B}$, which maps $\bullet$ to an object $\hat{d} \in \mathcal{B}$ such that $P(\hat{d}) = d$. Thus, $P$ is surjective on objects.

Suppose first that $c$ is in the image of $i$. Then (because $i$ is injective on objects) we can lift $c$ uniquely to an object of $A$, then map the result to $\mathcal{B}$ with $F$. We define this to be $\ell(c)$.

\begin{equation*}
  \begin{tikzcd}
    \mathcal{A}
    \arrow[r, "F"]
    \arrow[d, swap, "i"]
    & \mathcal{B}
    \arrow[d, "P"]
    \\
    \mathcal{C}
    \arrow[r, swap, "G"]
    \arrow[ur, dashed, "\ell"]
    & \mathcal{D}
  \end{tikzcd}
\end{equation*}

This makes both triangles formed commute trivially.

If $c$ is not mapped to by $i$, then we define $\ell(c)$ to be any lift $\widehat{G(c)}$. This makes the lower triangle commute by definition, and the upper triangle is irrelevant since $i$ never hits these objects.

Since $P$ is an equivalence of categories, the maps
\begin{equation*}
  P_{b, b'}\colon \mathcal{B}(b, b') \to \mathcal{D}(P(b),P(b'));\qquad \left[ b \overset{f}{\to} b' \right] \mapsto \left[ P(b) \overset{P(f)}{\to} P(b') \right]
\end{equation*}
are bijections, hence can be inverted. Thus we can define $\ell$ on morphisms $f\colon c \to c'$ in $\mathcal{C}$ by
\begin{equation*}
  \ell(f) = P^{-1}_{\ell(c), \ell(c')}(G(f)).
\end{equation*}

The upper triangle
\begin{equation*}
  \begin{tikzcd}
    \left[ a \overset{f}{\to} a' \right]
    \arrow[r, mapsto]
    \arrow[d, mapsto]
    & \left[ F(a)\overset{F(f)}{\to} F(a') \right] \overset{!}{=} \left[ \ell(i(a))\overset{\ell(i(f))}{\to}\ell(i(b)) \right]
    \\
    \left[ i(a)\overset{i(f)}{\to} i(a') \right]
    \arrow[ur, mapsto]
  \end{tikzcd}
\end{equation*}
commutes since $\ell(i(a)) = F(a)$ and $\ell(i(a')) = F(a')$, so
\begin{align*}
  \ell(i(f)) &= P^{-1}_{\ell(i(a)), \ell(i(a'))}(G(i(f))) \\
  &= P^{-1}_{F(a),F(a')}(P(F(f))) \\
  &= F(f).
\end{align*}

To see that the lower triangle
\begin{equation*}
  \begin{tikzcd}
    & \left[ \ell(c) \overset{\ell(f)}{\to} \ell(c') \right]
    \arrow[d, mapsto]
    \\
    \left[ c \overset{f}{\to} c' \right]
    \arrow[ur, mapsto]
    \arrow[r, mapsto]
    & \left[ G(c) \overset{G(f)}{\to} G(c') \right] \overset{!}{=} \left[ P(\ell(c)) \overset{P(\ell(f))}{\to} P(\ell(c')) \right]
  \end{tikzcd}
\end{equation*}
note that we have already seen that $P(\ell(c)) = G(c)$ and $P(\ell(c')) = G(c')$, and for $f\colon c \to c'$ we defined
\begin{equation*}
  \ell(f) = P^{-1}_{\ell(c), \ell(c')}(G(f)),
\end{equation*}
so
\begin{equation*}
  P(\ell(f)) = G(f).
\end{equation*}

Now we are in the situation where $i$ is both injective and an equivalence of categories, and $P$ is an isofibration.

Let $c \in \mathcal{C}$. Becase $i$ is essentially surjective we can always pick $c' \in \mathcal{C}$ such that $c'$ is in the image of $i$, i.e.\ such that there exists some $a_{c}$ with $i(a_{c}) = c' \cong c$. For each such $c$, pick such a $c'$ and $a_{c}$. If $c$ is already in the image of $i$, pick $c' = c$; that is, demand that $a_{i(a)} = a$.

This allows us to define, for each $c \in \mathcal{C}$ functors as follows.
\begin{equation*}
  \begin{tikzcd}
    *
    \arrow[r, mapsto]
    & F(a_{c})
    \\
    (* \leftrightarrow \bullet)
    \arrow[r, mapsto]
    & G(c') \leftrightarrow G(c)
  \end{tikzcd}
\end{equation*}
Then $P$'s isofibricity gives a functor $I \to \mathcal{B}$ making the diagram

commute, which sends $\bullet$ to some object $b_{c} \in \mathcal{B}$. By the commutativity, $P(b_{c}) = G(c)$.

Define the lift $\ell$ by $\ell(c) = b_{c}$. The upper triangle is
\begin{equation*}
  \begin{tikzcd}
    a
    \arrow[d, mapsto]
    \arrow[r, mapsto]
    & F(a) \overset{!}{=} \ell(i(a))
    \\
    i(a)
    \arrow[ur, mapsto]
  \end{tikzcd}
\end{equation*}
which commutes because $\ell(i_{a}) = F(a_{a}) = F(a)$.

The lower triangle
\begin{equation*}
  \begin{tikzcd}
    & \ell(c)
    \arrow[d, mapsto]
    \\
    c
    \arrow[r, mapsto]
    \arrow[ur, mapsto]
    & G(c) \overset{!}{=} P(\ell(c))
  \end{tikzcd}
\end{equation*}
commutes because $P(\ell(c)) = P(b_{c}) = G(c)$.

By the same logic as the previous problem, any morphism $f\colon c \to c'$ lifts well-definedly to a map $\hat{f}\colon a_{c} \to a_{c'}$. We define
\begin{equation*}
  \ell(f) = F(\hat{f}).
\end{equation*}
The upper triangle
\begin{equation*}
  \begin{tikzcd}
    f
    \arrow[r, mapsto]
    \arrow[d, mapsto]
    & F(f) \overset{!}{=} \ell(i(f))
    \\
    i(f)
    \arrow[ur, mapsto]
  \end{tikzcd}
\end{equation*}
commutes since $\ell(i(f)) = f$ by definition of $\hat{f}$. The lower triangle
\begin{equation*}
  \begin{tikzcd}
    & \ell(f)
    \arrow[d, mapsto]
    \\
    f
    \arrow[ur, mapsto]
    \arrow[r, mapsto]
    & P(\ell(f)) \overset{!}{=} G(f)
  \end{tikzcd}
\end{equation*}
commutes because
\begin{align*}
  P(\ell(f)) &= P(F(\hat{f})) \\
  &= G(i(\hat{f})) \\
  &= G(f).
\end{align*}

\subsubsection{Axiom M5: Factorization}
For any functor $F\colon \mathcal{C} \to \mathcal{D}$, define a functor $G\colon \mathcal{C} \to \mathcal{L}$ on objects by
\begin{equation*}
  x \mapsto (x, Fx, \id_{Fx}),
\end{equation*}
and on morphisms by $f \mapsto f$. The functor $G'$ sending
\begin{equation*}
  (c, d, f) \mapsto c.
\end{equation*}
is a weak inverse to $G$. As a check: these form an equivalence of categories since
\begin{equation*}
  G' \circ G\colon x \mapsto (x, Fx, \id_{Fx}) \mapsto x
\end{equation*}
and
\begin{equation*}
  G \circ G'\colon (c, d, f) \mapsto c \mapsto (c, Fc, \id_{Fc});
\end{equation*}
the map
\begin{equation*}
  G(\id_{c})\colon (c, d, f) \to (c, Fc, \id_{Fc})
\end{equation*}
is a natural isomorphism since the naturality square
\begin{equation*}
  \begin{tikzcd}
    (c, d, f)
    \arrow[r, "G(\id_{c})"]
    \arrow[d, swap, "g"]
    & (c, Fc, \id_{Fc})
    \arrow[d, "g"]
    \\
    (c', d', f')
    \arrow[r, swap, "G(\id_{c'})"]
    & (c', Fc', \id_{Fc'})
  \end{tikzcd}
\end{equation*}
commutes.

It is also clear that $G$ is injective on objects since $(c, d, f) \neq (c', d', f') \implies c \neq c'$, so $G \in \mathfrak{C}$.

Define also a functor $H\colon \mathcal{L} \to \mathcal{D}$ sending $(c, d, f) \mapsto d$ and $g\colon c \to c'$ to $F(g)$. Then
\begin{equation*}
  H \circ G\colon c \mapsto (c, Fc, \id_{Fc}) \mapsto Fc;\qquad g \mapsto g \mapsto Fg.
\end{equation*}
It remains only to show that $H \in \mathfrak{F}$, we need to show that it is an isofibration. Consider the following diagram.
\begin{equation*}
  \begin{tikzcd}
    (*)
    \arrow[ddd, swap, hookrightarrow]
    \arrow[rrr]
    &&& \mathcal{L}
    \arrow[ddd, "H"]
    \\
    & *
    \arrow[r, mapsto]
    \arrow[d, mapsto]
    & (c, d, f)
    \arrow[d, mapsto]
    \\
    & *
    \arrow[r, mapsto]
    & d
    \\
    (* \leftrightarrow \bullet)
    \arrow[rrr]
    &&& \mathcal{D}
  \end{tikzcd}
\end{equation*}
We need to show that given an isomorphism $r\colon d \to d'$ and an object $(c, d, f)$ in $\mathcal{L}$, we can find an object $(c', d', f')$ which is isomorphic to $(c, d, f)$. We can; taking $c' = c$, and $f' = r \circ f$ gives us what we want since the following diagram commutes.
\begin{equation*}
  \begin{tikzcd}
    & (c, d, f) \leftrightarrow (c, d', r \circ f)
    \arrow[d, mapsto]
    \\
    (* \leftrightarrow \bullet)
    \arrow[ur, mapsto]
    \arrow[r, mapsto]
    & c \leftrightarrow c'
  \end{tikzcd}
\end{equation*}

\textbf{Part 2:}
We define $G$ to be the canonical injection, and $H$ to be defined on objects as follows.
\begin{equation*}
  H\colon x \mapsto
  \begin{cases}
    Fx, &x \in \mathcal{C} \\
    x, &x \in \mathcal{D}
  \end{cases}
\end{equation*}
The hom-sets in $\mathcal{R}$ are already appropriate hom-sets in $\mathcal{D}$, so we don't need to do anything to define $H$ on morphisms: we simply define it to be the identity on hom-sets.

To see that $H$ is an equivalence of categories, note that it is essentially surjective because it is surjective `on the nose,' and that it is fully faithful because $H$ is the identity on hom-sets.

It is obvious that $G$ is injective on objects.

Now consider the following diagram.
\begin{equation*}
  \begin{tikzcd}
    (*)
    \arrow[r, "a"]
    \arrow[d]
    & \mathcal{R}
    \arrow[d]
    \\
    (* \leftrightarrow \bullet)
    \arrow[r, "b"]
    & \mathcal{D}
  \end{tikzcd}
\end{equation*}
Define a functor $\ell\colon I \to \mathcal{R}$ sending $* \mapsto a(*)$ and $\bullet \mapsto b(\bullet)$. This makes both triangles commute: the upper triangle is
\begin{equation*}
  \begin{tikzcd}
    *
    \arrow[r, mapsto]
    \arrow[d, mapsto]
    & a(*)
    \\
    *
    \arrow[ur, mapsto]
  \end{tikzcd}
\end{equation*}
and the lower triangle is
\begin{equation*}
  \begin{tikzcd}
    & a(*) \leftrightarrow b(\bullet)
    \\
    * \leftrightarrow \bullet
  \end{tikzcd}
\end{equation*}


\section{The Homotopy category of a model category}
\label{sec:the_homotopy_category_of_a_model_category}

In \hyperref[ssc:left_homotopy]{Subsection~\ref*{ssc:left_homotopy}}, we define the notion of a \emph{left homotopy} of maps $A \to B$, denoted $\overset{l}{\sim}$, and show that for any cofibrant object $A$ and any object $B$, left homotopy yields an equivalence relation on $\mathcal{C}(A, B)$.
\begin{equation*}
  \begin{tikzcd}
    \overbrace{A}^{\cof}
    \arrow[r, "f"]
    & B
  \end{tikzcd}
\end{equation*}

In \hyperref[ssc:right_homotopy]{Subsection~\ref*{ssc:right_homotopy}}, we define a dual notion, \emph{right homotopy,} denoted $\overset{r}{\sim}$, and show that for any object $A$ and any fibrant object $B$, right homotopy yields an equivalence relation on $\mathcal{C}(A, B)$.
\begin{equation*}
  \begin{tikzcd}
    A
    \arrow[r, "f"]
    & \overbrace{B}^{\fib}
  \end{tikzcd}
\end{equation*}

In \hyperref[ssc:homotopy_equivalents]{Subsection~\ref*{ssc:homotopy_equivalents}}, we show that when looking at maps from a cofibrant object to a fibrant object
\begin{equation*}
  \begin{tikzcd}
    \overbrace{A}^{\cof}
    \arrow[r, "f"]
    & \overbrace{B}^{\fib}
  \end{tikzcd}
\end{equation*}
a left homotopy $f \overset{l}{\sim} f'$ gives rise to a right homotopy and vice versa, and thus the two notions agree. Thus, for maps from a cofibrant object to a fibrant object, we do not keep track of left- and right homotopies, calling them simply \emph{homotopies.}

This allows us to define, for two maps between fibrant-cofibrant objects, a notion of homotopy equivalence; we say that a map $f\colon A \to B$, where $A$ and $B$ are fibrant-cofibrant, is a homotopy equivalence if it has a homotopy inverse $g$, i.e.\ a map $g$ such that
\begin{equation*}
  f \circ g \sim \id_{B},\qquad \text{and}\qquad g \circ f \sim \id_{A}.
\end{equation*}

Next, we show that this notion is preserved under composition: if $f$ and $g$ are homotopy equivalences, then so is $g \circ f$. This allows us to prove Whitehead's theorem, which tells us that any weak equivalence between fibrant-cofibrant objects is a homotopy equivalence.

Finally, we define the homotopy category $\mathrm{Ho}(\mathcal{C})$ of a model category, whose objects are the fibrant-cofibrant objects of $\mathcal{C}$, and whose morphisms $X \to Y$ are the homotopy classes of morphisms $X \to Y$.

\subsection{Left homotopy}
\label{ssc:left_homotopy}

\begin{definition}[cylinder object]
  \label{def:cylinder_object}
  Let $\mathcal{C}$ be a model category, and let $A \in \mathcal{C}$. A \defn{cylinder object for $A$} is an object $C$ together with a factorization
  \begin{equation*}
    \begin{tikzcd}
      A \amalg A
      \arrow[r, "{(\id, \id)}"]
      \arrow[d, swap, "{i = (i_{0}, i_{1})}"]
      & A
      \\
      C
      \arrow[ur, swap, "\sigma"]
    \end{tikzcd}
  \end{equation*}
  such that $i$ is a cofibration and $\sigma$ is a weak equivalence.
\end{definition}

In any model category, every object $A$ admits a cylinder object. \hyperref[item:MC_morphisms_factorize]{Axiom~\ref*{item:MC_morphisms_factorize}} guarantees that we can find an object $C$ and maps $i$ and $c$ as above where $i$ is a cofibration and $\sigma$ is a trivial fibration, hence certainly a fibration. However, many model categories have an object such $I$ such that any object $A$ has a cylinder object built from $A$ and $I$, for example by taking the product $A \times I$.

\begin{example}
  \label{eg:cylinder_objects_in_top}
  In the category $\Top$ with the Quillen model structure of \hyperref[eg:model_categories]{Example~\ref*{eg:model_categories}}, for each object $X$ there is a cylinder object $X \times I$, where $I$ is the unit interval $[0, 1]$.
\end{example}

\begin{example}
  \label{eg:cylinder_objects_in_sset}
  In $\SSet$ equipped with the Kan model structure, for each simplicial set $S$ there is a cylinder object, given by $\Delta^{1} \times S$.
  \begin{equation*}
    \begin{tikzcd}
      S \amalg S
      \arrow[r, "{(\id, \id)}"]
      \arrow[d, swap, "{(s_{0} \times \id, s_{1} \times \id)}"]
      & S
      \\
      \Delta^{1} \times S
      \arrow[ur, swap, "\pi_{2}"]
    \end{tikzcd}
  \end{equation*}

  In this case, $i$ is a cofibration because it is an inclusion (hence a monomorphism as required) and $\pi_{2}$ is a weak equivalence (i.e.\ a homotopy equivalence) with homotopy inverse $s \mapsto (0, s)$.
\end{example}

\begin{example}
  \label{eg:cylinder_objects_in_model_structure_on_cat}
  In the model category on categories, defined in \hyperref[ssc:a_model_structure_on_the_category_of_categories]{Subsection~\ref*{ssc:a_model_structure_on_the_category_of_categories}}, any category $\mathcal{C}$ has cylinder object $\mathcal{C} \times I$, taken with the below factorization.
  \begin{equation*}
    \begin{tikzcd}
      \mathcal{C} \amalg \mathcal{C}
      \arrow[d, swap, "i = {(\id \times \{*\}, \id \times \{\bullet\})}"]
      \arrow[r, "{(\id, \id)}"]
      & \mathcal{C}
      \\
      \mathcal{C} \times I
      \arrow[ur, swap, "\pi"]
    \end{tikzcd}
  \end{equation*}
  The map $i$ is obviously injective on objects, and the map $\pi$ is an equivalence since the map $c \mapsto (c, 0)$ is a weak inverse.
\end{example}

\begin{example}
  Let $R$ be a small ring, and consider the category $\mathbf{Ch}_{\geq}(R)$ with the projective model structure of \hyperref[ssc:a_model_structure_on_the_category_of_categories]{Subsection~\ref*{ssc:a_model_structure_on_the_category_of_categories}}. Denote by $I_{\bullet}$ the chain complex\footnote{This is the chain complex corresponding, under the Dold-Kan correspondence, to the unit interval.}
  \begin{equation*}
    \begin{tikzcd}
      \cdots
      \arrow[r]
      & 0
      \arrow[r]
      & R
      \arrow[r, "{( \id, -\id )}"]
      & R \oplus R
    \end{tikzcd}.
  \end{equation*}
  Then each chain complex $A_{\bullet}$ has a cylinder object
  \begin{equation*}
    \begin{tikzcd}
      A \oplus A
      \arrow[d]
      \arrow[r, "{(\id, \id)}"]
      & A
      \\
      A \otimes I
      \arrow[ur]
    \end{tikzcd}.
  \end{equation*}
\end{example}

\begin{definition}[left homotopy]
  \label{def:left_homotopy}
  Let $\mathcal{C}$ be a model category, and let $f$, $g\colon A \to B$ be two morphsims in $\mathcal{C}$. A \defn{left homotopy} between $f$ and $g$ consists of a cylinder object $C$ for $A$ (\hyperref[def:cylinder_object]{Definition~\ref*{def:cylinder_object}}) and a map $H\colon C \to B$ such that the following diagram commutes.
  \begin{equation*}
    \begin{tikzcd}
      A \amalg A
      \arrow[d, swap, "i"]
      \arrow[r, "{(f, g)}"]
      & B
      \\
      C
      \arrow[ur, swap, "H"]
    \end{tikzcd}
  \end{equation*}

  In this case, we will write
  \begin{equation*}
    f \overset{l}{\sim} g.
  \end{equation*}
\end{definition}

\begin{example}
  In the projective model structure on $\mathbf{Ch}_{\geq 0}(R)$, a left homotopy with respect to the cylinder object of \hyperref[eg:cylinder_objects_in_model_structure_on_cat]{Example~\ref*{eg:cylinder_objects_in_model_structure_on_cat}} is a chain homotopy.
\end{example}

\begin{example}
  In the Quillen model structure on $\Top$ with cylinder objects as in \hyperref[eg:cylinder_objects_in_top]{Example~\ref*{eg:cylinder_objects_in_top}}, a left homotopy between $f$ and $g$ is a homotopy in the standard sense.
\end{example}

\begin{example}
  In the Kan model structure on $\SSet$ with cylinder objects as in \hyperref[eg:cylinder_objects_in_sset]{Example~\ref*{eg:cylinder_objects_in_sset}}, left homotopy reproduces \hyperref[def:homotopy]{Definition~\ref*{def:homotopy}}.
\end{example}

\begin{lemma}
  \label{lemma:cylinder_injections_are_trivial_cofibrations}
  Let $A$ be a cofibrant object. If $C$ is a cylinder object for $A$, then the maps $i_{1}$ and $i_{2}$ are trivial cofibrations.
\end{lemma}
\begin{proof}
  Since $A$ is cofibrant, by definition the map $\emptyset \to A$ is a cofibration. Consider the following pushout square.
  \begin{equation*}
    \begin{tikzcd}
      \emptyset
      \arrow[r]
      \arrow[d]
      & A
      \arrow[d]
      \\
      A
      \arrow[r]
      & A \amalg A
    \end{tikzcd}
  \end{equation*}

  Since by \hyperref[cor:trivial_fibrations_and_cofibrations_are_saturated]{Corollary~\ref*{cor:trivial_fibrations_and_cofibrations_are_saturated}} cofibrations are closed under pushout, the canonical injections $\iota_{\alpha}\colon A \to A \amalg A$, $\alpha = 1$, $2$, are also cofibrations. Since $i_{\alpha} = i \circ \iota_{\alpha}$ is a composition of cofibrations, it is again a cofibration by closure under countable composition.

  Since the diagram
  \begin{equation*}
    \begin{tikzcd}[row sep=large, column sep=large]
      A
      \arrow[d, hookrightarrow, "\iota_{\alpha}"]
      \arrow[rd, "\id"]
      \arrow[dd, bend right, swap, "\iota_{\alpha}"]
      \\
      A \amalg A
      \arrow[r, "{(\id, \id)}"]
      \arrow[d, "i"]
      & A
      \\
      C
      \arrow[ur, swap, "\sigma"]
    \end{tikzcd}
  \end{equation*}
  commutes, we can write $\id_{A} = \sigma \circ i_{\alpha}$ (with $\id_{A}$ a weak equivalence because it is an isomorphsim and $\sigma$ a weak equivalence by assumption), so by the two-out-of-three law $i_{\alpha}$ is a weak equivalence.
\end{proof}

\begin{proposition}
  \label{prop:left_homotopy_is_equivalence_relation}
  If $A$ is a cofibrant object, then left homotopy of maps $A \to B$ is an equivalence relation.
\end{proposition}
\begin{proof}
  To see that left homotopy is an equivalence relation, let $C$ be any cylinder object over $A$.
  \begin{equation*}
    \begin{tikzcd}
      A \amalg A
      \arrow[r, "{(\id, \id)}"]
      \arrow[d, swap, "{i = (i_{0}, i_{1})}"]
      & A
      \\
      C
      \arrow[ur, swap, "\sigma"]
    \end{tikzcd}
  \end{equation*}
  Then the following diagram commutes.
  \begin{equation*}
    \begin{tikzcd}[column sep=large]
      A \amalg A
      \arrow[r, "{(f, f)}"]
      \arrow[d, swap, "i"]
      & B
      \\
      C
      \arrow[ur, swap, "f \circ \sigma"]
    \end{tikzcd}
  \end{equation*}
  This means that $f \circ \sigma$ is a left homotopy from $f$ to itself.

  Next, we show that left homotopy is symmetric. First, we note that in any category with coproducts, there is a natural isomorphism $\mathrm{swap}_{A}\colon A \amalg A \to A \amalg A$ which switches the components. Because $\cof$ is saturated and thus contains all isomorphisms, $\mathrm{swap}_{A}$ is a cofibration. This gives us a new factorization exhibiting $C$ as a cylinder object of $A$.
  \begin{equation*}
    \begin{tikzcd}
      A \amalg A
      \arrow[r, "{(\id, \id)}"]
      \arrow[d, swap, "i \circ \mathrm{swap}_{A}"]
      & A
      \\
      C
      \arrow[ur, swap, "\sigma"]
    \end{tikzcd}
  \end{equation*}
  Let $H$ be a left homotopy $f \overset{l}{\sim} g$ as follows.
  \begin{equation*}
    \begin{tikzcd}
      A \amalg A
      \arrow[r, "{(f, g)}"]
      \arrow[d, swap, "i"]
      & B
      \\
      C
      \arrow[ur, swap, "H"]
    \end{tikzcd}
  \end{equation*}

  The diagram
  \begin{equation*}
    \begin{tikzcd}[column sep=large]
      A \amalg A
      \arrow[d, swap, "\mathrm{swap}_{A}"]
      \arrow[dr, "{(g, f)}"]
      \\
      A \amalg A
      \arrow[r, "{(f, g)}"]
      \arrow[d, swap, "i"]
      & B
      \\
      C
      \arrow[ur, swap, "H"]
    \end{tikzcd}
  \end{equation*}
  exhibits a left homotopy $g \overset{l}{\sim} f$.

  It remains only to show that left homotopy is transitive. To this end, suppose $f \overset{l}{\sim} g$ and $g \overset{l}{\sim} h$,
  \begin{equation*}
    \begin{tikzcd}
      A \amalg A
      \arrow[r, "{(f, g)}"]
      \arrow[d, swap, "{(i_{0}, i_{1})}"]
      & B
      \\
      C
      \arrow[ur, swap, "H"]
    \end{tikzcd}
    \qquad
    \begin{tikzcd}
      A \amalg A
      \arrow[r, "{(g, h)}"]
      \arrow[d, swap, "{(i_{0}', i_{1}')}"]
      & B
      \\
      C'
      \arrow[ur, swap, "H'"]
    \end{tikzcd}
  \end{equation*}
  where $C$ and $C'$ are cylinder objects for $A$ with the following factorizations.
  \begin{equation*}
    \begin{tikzcd}
      A \amalg A
      \arrow[r, "{(\id, \id)}"]
      \arrow[d, swap, "{(i_{0}, i_{1})}"]
      & A
      \\
      C
      \arrow[ur, swap, "\sigma"]
    \end{tikzcd}
    \qquad
    \begin{tikzcd}
      A \amalg A
      \arrow[r, "{(\id, \id)}"]
      \arrow[d, swap, "{(i_{0}', i_{1}')}"]
      & B
      \\
      C'
      \arrow[ur, swap, "\sigma'"]
    \end{tikzcd}
  \end{equation*}

  Consider the following diagram, where the top left square is a pushout square.
  \begin{equation*}
    \begin{tikzcd}
      A
      \arrow[r, "i_{0}'"]
      \arrow[d, swap, "i_{1}"]
      & C'
      \arrow[d, swap, "j_{1}"]
      \arrow[ddr, bend left, "H'"]
      \\
      C
      \arrow[r, "j_{0}'"]
      \arrow[drr, swap, bend right, "H"]
      & C \coprod_{A} C'
      \arrow[dr, dashed, "\exists!\tilde{H}"]
      \\
      && B
    \end{tikzcd}
  \end{equation*}

  The following diagram commutes.
  \begin{equation*}
    \begin{tikzcd}
      A \amalg A
      \arrow[r, "{(f, h)}"]
      \arrow[d, swap, "{(j_{0}' \circ i_{0}, j_{1} \circ i_{0}')}"]
      & B
      \\
      C \amalg_{A} C'
      \arrow[ur, swap, "\tilde{H}"]
    \end{tikzcd}
  \end{equation*}

  It remains only to show that $C \amalg_{A} C'$ is a cylinder object for $A$. Define a map $\tilde{\sigma}$ as follows.
  \begin{equation*}
    \begin{tikzcd}
      A
      \arrow[r, "i_{0}'"]
      \arrow[d, swap, "i_{1}"]
      & C'
      \arrow[d, swap, "j_{1}"]
      \arrow[ddr, bend left, "\sigma'"]
      \\
      C
      \arrow[r, "j_{0}'"]
      \arrow[drr, swap, bend right, "\sigma"]
      & C \coprod_{A} C'
      \arrow[dr, dashed, "\exists!\tilde{\sigma}"]
      \\
      && A
    \end{tikzcd}
  \end{equation*}
  Since $j_{1} \in \W \cap \cof$ and $\sigma' \in \W$, we have by \hyperref[item:MC_two_out_of_three]{Axiom~\ref*{item:MC_two_out_of_three}} that $\tilde{\sigma}$ is a weak equivalence.

  Thus, to show that $C \amalg_{A} C'$ is a cylinder object with the factorization
  \begin{equation*}
    \begin{tikzcd}
      A \amalg A
      \arrow[r, "{(\id, \id)}"]
      \arrow[d, swap, "{(j_{0}' \circ i_{0}, j_{1} \circ i_{0}')}"]
      & A
      \\
      C \amalg_{A} C'
      \arrow[ur, swap, "\tilde{\sigma}"]
    \end{tikzcd}
  \end{equation*}
  we need only show that $(j'_{0} \circ i_{1}, j_{1} \circ i'_{0})$ is a cofibration.

  First note that the following square is a pushout square, where $\delta$ is the codiagonal.
  \begin{equation*}
    \begin{tikzcd}[row sep=huge, column sep=huge]
      (A \amalg A) \amalg (A \amalg A)
      \arrow[r, "{(i_{0}, i_{1}) \amalg (i'_{0}, i'_{1})}"]
      \arrow[d, swap, "{(\id, \delta, \id)}"]
      & C \amalg C'
      \arrow[d, "{(j'_{0}, j_{1})}"]
      \\
      A \amalg A \amalg A
      \arrow[r]
      & C \amalg_{A} C
    \end{tikzcd}
  \end{equation*}
  To see this, note maps
  \begin{equation*}
    (\alpha, \beta)\colon C \amalg C' \to Z \qquad\text{and}\qquad (a,b,c)\colon A \amalg A \amalg A
  \end{equation*}
  which form a cocone satisfy $\beta \circ i'_{0} = b$ and $\alpha \circ i_{1} = f$.
  \begin{equation*}
    \begin{tikzcd}[row sep=huge, column sep=huge]
      (A \amalg A) \amalg (A \amalg A)
      \arrow[r, "{(i_{0}, i_{1}) \amalg (i'_{0}, i'_{1})}"]
      \arrow[d, swap, "{(\id, \delta, \id)}"]
      & C \amalg C'
      \arrow[d]
      \arrow[ddr, bend left, "{(\alpha, \beta)}"]
      \\
      A \amalg A \amalg A
      \arrow[r, "k"]
      \arrow[drr, bend right, swap, "{(a,b,c)}"]
      & C \amalg_{A} C
      \arrow[dr, dashed, "\exists!"]
      \\
      && Z
    \end{tikzcd}
  \end{equation*}
  Thus, there is a unique map
  \begin{equation*}
    C \amalg_{A} C \to Z
  \end{equation*}
  making the above diagram commute.

  Note that because $(i_{0}, i_{1}) \amalg (i'_{0}, i'_{1})$ is a cofibration, $k$ must be also.

  Also, the map $A \amalg A \to A \amalg A \amalg A$ is a cofibration because the following diagram is a pushout.
  \begin{equation*}
    \begin{tikzcd}
      \emptyset
      \arrow[r]
      \arrow[d]
      & A
      \arrow[d]
      \\
      A \amalg A
      \arrow[r]
      & A \amalg A \amalg A
    \end{tikzcd}
  \end{equation*}
  Thus, the composition
  \begin{equation*}
    A \amalg A \to A \amalg A \amalg A \to C \amalg_{A} C'
  \end{equation*}
  is a cofibration.
\end{proof}

\subsection{Right homotopy}
\label{ssc:right_homotopy}

There is a dual notion to left fibrations, called right fibrations. The theory of right fibrations is dual to that of left fibrations.

\begin{definition}[path object]
  \label{def:path_object}
  Let $B$ be an object of a model category. A \defn{path object} for $B$ is an object $P$, together with a factorization
  \begin{equation*}
    \begin{tikzcd}
      & P
      \arrow[d, "p"]
      \\
      B
      \arrow[ur, "s"]
      \arrow[r, swap, "{(\id, \id)}"]
      & B \times B
    \end{tikzcd}
  \end{equation*}
  where $p$ is a fibration and $s$ is a weak equivalence.
\end{definition}

\begin{definition}[right homotopy]
  \label{def:right_homotopy}
  A \defn{right homotopy} between morphisms $f$, $g\colon A \to B$ consists of a path object $P$ and a factorization as follows.
  \begin{equation*}
    \begin{tikzcd}
      & P
      \arrow[d, "p"]
      \\
      A
      \arrow[ur, "H"]
      \arrow[r, swap, "{(f, g)}"]
      & B \times B
    \end{tikzcd}
  \end{equation*}
\end{definition}

\begin{lemma}
  \label{lemma:path_projections_are_trivial_fibrations}
  Let $B$ be a fibrant object. If $P$ is a cylinder object for $B$, then the maps $p_{0}$ and $p_{1}$ are trivial fibrations.
\end{lemma}
\begin{proof}
  Apply \hyperref[lemma:cylinder_injections_are_trivial_cofibrations]{Lemma~\ref*{lemma:cylinder_injections_are_trivial_cofibrations}} to the opposite model category (\hyperref[eg:opposite_model_category]{Example~\ref*{eg:opposite_model_category}}).
\end{proof}

\begin{proposition}
  \label{prop:right_homotopy_is_equivalence_relation}
  Let $B$ a fibrant object. Right homotopy of maps $f$, $g\colon A \to B$ is an equivalence relation.
\end{proposition}
\begin{proof}
  Apply \hyperref[prop:left_homotopy_is_equivalence_relation]{Proposition~\ref*{prop:left_homotopy_is_equivalence_relation}} to the opposite model category (\hyperref[eg:opposite_model_category]{Example~\ref*{eg:opposite_model_category}}).
\end{proof}

\subsection{Homotopy equivalents}
\label{ssc:homotopy_equivalents}

Fix objects $A$ and $B$ in a model category, and consider maps $A \to B$. In this section we will show that under certain conditions, namely when $A$ and $B$ are both fibrant-cofibrant, left and right homotopy are equivalent.

\begin{lemma}
  \label{lemma:left_homotopy_implies_right_homotopy}
  Suppose $A$ is cofibrant, and suppose that $f$, $g\colon A \to B$ are left homotopic. Then given any path object $P$ for $B$, there is a right homotopy between $f$ and $g$ with underlying path object $P$.
\end{lemma}
\begin{proof}
  Suppose we are given a left homotopy
  \begin{equation*}
    \begin{tikzcd}
      A \amalg A
      \arrow[r, "{(f, g)}"]
      \arrow[d, swap, "i"]
      & B
      \\
      C
      \arrow[ur, swap, "H"]
    \end{tikzcd}
  \end{equation*}
  and suppose that $P$ has the following factorization
  \begin{equation*}
    \begin{tikzcd}
      & P
      \arrow[d, "p"]
      \\
      B
      \arrow[ur, "s"]
      \arrow[r, swap, "{(\id, \id)}"]
      & B \times B
    \end{tikzcd}
  \end{equation*}
  Consider the following diagram.
  \begin{equation*}
    \begin{tikzcd}
      A
      \arrow[r, "s \circ f"]
      \arrow[d, swap, "i_{0}"]
      & P
      \arrow[d, "p"]
      \\
      C
      \arrow[r, swap, "{(f \circ \sigma, H)}"]
      & B \times B
    \end{tikzcd}
  \end{equation*}
  It commutes because
  \begin{equation*}
    p \circ s \circ f = (\id, \id) \circ f = (f, f)
  \end{equation*}
  and
  \begin{equation*}
    (f \circ \sigma, H) \circ i_{0} = (f \circ \sigma \circ i_{0}, H \circ i_{0}) = (f \circ \id, f) = (f, f).
  \end{equation*}

  By definition, $p$ is a fibration, and we have seen in \hyperref[lemma:cylinder_injections_are_trivial_cofibrations]{Lemma~\ref*{lemma:cylinder_injections_are_trivial_cofibrations}} that $i_{0}$ is a trivial cofibration. Thus, we have a lift $h\colon C \to P$.
  \begin{equation*}
    \begin{tikzcd}
      A
      \arrow[r, "s \circ f"]
      \arrow[d, swap, "i_{0}"]
      & P
      \arrow[d, "p"]
      \\
      C
      \arrow[r, swap, "{(f \circ \sigma, H)}"]
      \arrow[ur, dashed, "h"]
      & B \times B
    \end{tikzcd}
  \end{equation*}

  Then we have
  \begin{align*}
    p \circ h \circ i_{1} &= (f \circ \sigma, H) \circ i_{1} \\
    &= (f \circ \sigma \circ i_{1}, H \circ i_{1}) \\
    &= (f, g),
  \end{align*}
  so the diagram
  \begin{equation*}
    \begin{tikzcd}
      & P
      \arrow[d, "p"]
      \\
      A
      \arrow[ur, "h \circ i_{1}"]
      \arrow[r, swap, "{(f, g)}"]
      & B \times B
    \end{tikzcd}
  \end{equation*}
  commutes. Thus, $h \circ i_{1}$ gives a right homotopy between $f$ and $g$.
\end{proof}

\begin{corollary}
  Let $f$, $g\colon A \to B$ be morphisms, and suppose that $A$ is cofibrant and $B$ is fibrant. Then the following are equivalent.
  \begin{enumerate}
    \item $f$ and $g$ are left homotopic.

    \item For every path object $P$ for $B$, there is a right homotopy between $f$ and $g$ with path object $P$.

    \item $f$ and $g$ are right homotopic.

    \item For every cylinder object $C$ for $A$, there is a left homotopy between $f$ and $g$ with cylinder object $C$.
  \end{enumerate}
\end{corollary}
\begin{proof}
  \leavevmode
  \begin{itemize}
    \item[$1 \Rightarrow 2$.] \hyperref[lemma:left_homotopy_implies_right_homotopy]{Lemma~\ref*{lemma:left_homotopy_implies_right_homotopy}}.

    \item[$2 \Rightarrow 3$.] Trivial.

    \item[$3 \Rightarrow 4$.] Dual to \hyperref[lemma:left_homotopy_implies_right_homotopy]{Lemma~\ref*{lemma:left_homotopy_implies_right_homotopy}}.

    \item[$4 \Rightarrow 1$.] Trivial.
  \end{itemize}
\end{proof}

The above result shows that when considering morphisms between a cofibrant object and a fibrant object, there is no difference between left and right fibrations. Therefore, we will not distinguish them.

\begin{definition}[homotopy equivalence]
  \label{def:homotopy_equivalence_model_category}
  Let $f\colon X \to Y$ be a morphism with $X$ and $Y$ fibrant-cofibrant. We say that $f$ is a \defn{homotopy equivalence} if there exists a morphism $g\colon Y \to X$ such that $g \circ f \sim \id_{X}$ and $f \circ g \sim \id_{Y}$.
\end{definition}

The next lemma shows that given a homotopy equivalence, the result of pre- or post-composing with any morphism is again a homotopy equivalence.
\begin{lemma}
  \label{lemma:composition_with_any_morphism_preserves_homotopy_equivalence}
  Let $f$, $g\colon X \to Y$ be homotopy equivalent morphisms between fibrant-cofibrant objects, and let $W$ and $Z$ be fibrant-cofibrant. Then the following hold.
  \begin{enumerate}
    \item For any $h\colon Y \to Z$, $h \circ f \sim h \circ g$.

    \item For any $h'\colon W \to X$, $f \circ h' \sim g \circ h'$.
  \end{enumerate}
\end{lemma}
\begin{proof}
  \leavevmode
  \begin{enumerate}
    \item Suppose our homotopy is given by the following factorization.
      \begin{equation*}
        \begin{tikzcd}
          X \amalg X
          \arrow[d, swap, "i"]
          \arrow[r, "{(f, g)}"]
          & Y
          \\
          C
          \arrow[ur, swap, "H"]
        \end{tikzcd}
      \end{equation*}
      Post-composing with $h$ gives us the following
      \begin{equation*}
        \begin{tikzcd}
          X \amalg X
          \arrow[d, swap, "i"]
          \arrow[r, "{(f, g)}"]
          & Y
          \arrow[r, "h"]
          & Z
          \\
          C
          \arrow[ur, swap, "H"]
        \end{tikzcd}
        \qquad \implies
        \begin{tikzcd}
          X \amalg X
          \arrow[d, swap, "i"]
          \arrow[r, "{(h \circ f, h \circ g)}"]
          & Z
          \\
          C
          \arrow[ur, swap, "h \circ H"]
        \end{tikzcd}
      \end{equation*}
      But the second diagram is precisely a homotopy $h \circ f \sim h \circ g$.

    \item Now suppose our homotopy is given by the following factorization.
      \begin{equation*}
        \begin{tikzcd}
          & P
          \arrow[d, "s"]
          \\
          X
          \arrow[r, swap, "{(f, g)}"]
          \arrow[ur, "H'"]
          & Y \times Y
        \end{tikzcd}
      \end{equation*}

      Pre-composing with $h$ yields the following.
      \begin{equation*}
        \begin{tikzcd}
          && P
          \arrow[d, "s"]
          \\
          W
          \arrow[r, "h"]
          & X
          \arrow[r, swap, "{(f, g)}"]
          \arrow[ur, "H'"]
          & Y \times Y
        \end{tikzcd}
        \qquad\implies\qquad
        \begin{tikzcd}
          & P
          \arrow[d, "s"]
          \\
          X
          \arrow[r, swap, "{(f \circ h, g \circ h)}"]
          \arrow[ur, "H' \circ h"]
          & Y \times Y
        \end{tikzcd}
      \end{equation*}
      This is preciesly a homotopy $f \circ h \sim g \circ f$.
  \end{enumerate}
\end{proof}

\begin{corollary}
  \label{cor:composition_of_homotopy_equivalences_is_homotopy_equivalence}
  Let $X$, $Y$, and $Z$ be fibrant-cofibrant objects, and let $f$ and $g$ be homotopy equivalences as follows.
  \begin{equation*}
    \begin{tikzcd}
      X
      \arrow[r, "f"]
      & Y
      \arrow[r, "g"]
      & Z
    \end{tikzcd}
  \end{equation*}
  Then $g \circ f$ is a homotopy equivalence.
\end{corollary}
\begin{proof}
  By definition, $f$ has a homotopy inverse $f'$ such that $f' \circ f \sim \id$ and $f \circ f' \sim \id$. Similarly, there is a $g'$ such that $g' \circ g \sim \id$ and $g \circ g' \sim \id$. But then
  \begin{align*}
    \id &\sim g' \circ g \\
    &\sim f' \circ g' \circ g \\
    &\sim f' \circ g' \circ g \circ f
  \end{align*}
  and
  \begin{align*}
    \id &\sim f' \circ f \\
    &\sim g' \circ f' \circ f \circ g.
  \end{align*}
  so $f' \circ g'$ functions as a homotopy inverse to $g \circ f$.
\end{proof}

\begin{lemma}
  \label{lemma:pre_whitehead}
  Let $X$ and $Y$ be fibrant-cofibrant objects, and $f\colon X \to Y$ a weak equivalence.
  \begin{enumerate}
    \item If $f$ is a fibration, then it is a homotopy equivalence.

    \item If $f$ is a cofibration, then it is a homotopy equivalence.
  \end{enumerate}
\end{lemma}
\begin{proof}
  \leavevmode
  \begin{enumerate}
    \item We have a lift as follows.
      \begin{equation*}
        \begin{tikzcd}
          \emptyset
          \arrow[r]
          \arrow[d]
          & X
          \arrow[d, "f"]
          \\
          Y
          \arrow[r, swap, "\id"]
          \arrow[ur, dashed, "g"]
          & Y
        \end{tikzcd}
      \end{equation*}
      Thus, $f \circ g = \id_{Y}$.

      Thus the diagram
      \begin{equation*}
        \begin{tikzcd}[row sep=huge, column sep=huge]
          X \amalg X
          \arrow[r, "{(g \circ f, \id)}"]
          \arrow[d, swap, "i"]
          & X
          \arrow[d, "f"]
          \\
          C
          \arrow[r, swap, "f \circ \sigma"]
          & Y
        \end{tikzcd}
      \end{equation*}
      commutes, because
      \begin{align*}
        f \circ (g \circ f, \id) &= (f \circ g \circ f, f) \\
        &= (f, f) \\
        &= (f \circ \sigma \circ i_{1}, f \circ \sigma \circ i_{2}) \\
        &= f \circ \sigma \circ i.
      \end{align*}

      Since $f$ is a trivial fibration and $i$ is a cofibration, we have by \hyperref[item:MC_solve_some_lifting_problems]{Axiom~\ref*{item:MC_solve_some_lifting_problems}} a lift as follows.
      \begin{equation*}
        \begin{tikzcd}[row sep=large, column sep=large]
          X \amalg X
          \arrow[r, "{(g \circ f, \id)}"]
          \arrow[d, swap, "i"]
          & X
          \arrow[d, "f"]
          \\
          C
          \arrow[ur, dashed, "H"]
          \arrow[r, swap, "f \circ \sigma"]
          & Y
        \end{tikzcd}
      \end{equation*}
      The upper triangle tells us that $H$ is a left homotopy between $g \circ f$ and $\id_{X}$.

      Since $f \circ g = \id_{Y}$, the cylinder object $C$ itself functions as a homotopy $f \circ g \sim \id_{Y}$. Thus, $f$ is a homotopy equivalent.

    \item The commutativity of the solid diagram
      \begin{equation*}
        \begin{tikzcd}[row sep=large, column sep=large]
          X
          \arrow[r, "s \circ f"]
          \arrow[d, swap, "f"]
          & P
          \arrow[d, "p"]
          \\
          Y
          \arrow[r, swap, "{(\id_{Y}, f \circ h)}"]
          \arrow[ur, dashed]
          & Y \times Y
        \end{tikzcd}
      \end{equation*}
      together with the fact that $f$ is a trivial cofibration and $p$ is a fibration imply that there is a dashed lift. This exhibits a right homotopy $f \circ h \sim \id_{Y}$. But $h \circ f = \id_{X}$, so $h \circ f \sim \id_{X}$.

      So far we have shown that the result holds if $f$ is a trivial fibration or a trivial cofibration.

      Now suppose $f$ is merely a weak equivalence. Then we can factor
      \begin{equation*}
        f = g \circ h,
      \end{equation*}
      where $g$ is a fibration and $h$ is a trivial cofibration. But by \hyperref[item:MC_two_out_of_three]{Axiom~\ref*{item:MC_two_out_of_three}}, $g$ is also a trivial cofibration.
  \end{enumerate}
\end{proof}

\begin{theorem}[Whitehead]
  \label{thm:whitehead}
  Let $f\colon X \to Y$ be a weak equivalence, and suppose that $X$ and $Y$ are fibrant-cofibrant. Then $f$ is a homotopy equivalence.
\end{theorem}
\begin{proof}
  By \hyperref[item:MC_morphisms_factorize]{Axiom~\ref*{item:MC_morphisms_factorize}}, we are allowed to factor $f = p \circ i$, where $p$ is a fibration, $i$ is a cofibration, and at least one of $p$ and $i$ is a weak equivalence. By \hyperref[item:MC_two_out_of_three]{Axiom~\ref*{item:MC_two_out_of_three}}, both $i$ and $p$ are weak equivalences.
  \begin{equation*}
    \begin{tikzcd}
      X
      \arrow[rr, "f"]
      \arrow[dr, swap, "\W \cap \cof \ni i"]
      && Y
      \\
      & Z
      \arrow[ur, swap, "p \in \W \cap \fib"]
    \end{tikzcd}
  \end{equation*}

  By assumption, $X$ is cofibrant, i.e.\ the map $\emptyset \to X$ is a cofibration. The map $i\colon X \to Z$ is also a cofibration, so their composition must also be a cofibration. But this means that the map $\emptyset \to Z$ is also a cofibration, so $Z$ is cofibrant

  Dually, the map $Y \to *$ is a fibration. Precomposing by $p$, we have that $Z$ is fibrant. Thus, $i$ and $p$ are homotopy equivalences by \hyperref[lemma:pre_whitehead]{Lemma~\ref*{lemma:pre_whitehead}}, parts 2 and 1 respectively. But by \hyperref[lemma:composition_with_any_morphism_preserves_homotopy_equivalence]{Lemma~\ref*{lemma:composition_with_any_morphism_preserves_homotopy_equivalence}}, their composition is also a homotopy equivalence.
\end{proof}

\begin{definition}[homotopy category of a model category]
  \label{def:homopy_category}
  Let $\mathcal{C}$ be a model category. The \defn{homotopy category} of $\mathcal{C}$, denoted $\mathrm{Ho}(\mathcal{C})$, is defined as follows.
  \begin{itemize}
    \item The objects of $\mathrm{Ho}(\mathcal{C})$ are the fibrant-cofibrant objects of $\mathcal{C}$.

    \item The set of morphisms $X \to Y$ is the set of homotopy classes of $\mathcal{C}(X, Y)$.

    \item Composition is given by the rule
      \begin{equation*}
        [f] \circ [g] \equiv [f \circ g].
      \end{equation*}
      This is well-defined by \hyperref[lemma:composition_with_any_morphism_preserves_homotopy_equivalence]{Lemma~\ref*{lemma:composition_with_any_morphism_preserves_homotopy_equivalence}}.
  \end{itemize}
\end{definition}

\begin{example}
  Consider the category $\SSet$ together with the Kan model structure (\hyperref[eg:model_categories]{Example~\ref*{eg:model_categories}}). We have seen (in \hyperref[eg:cylinder_objects_in_sset]{Example~\ref*{eg:cylinder_objects_in_sset}}) that for any simplicial set $S$, the object $\Delta^{1} \times S$ is a cylinder object for $S$. Following our proofs through, one finds that the homotopy theory in the model structure on $\SSet$ agrees with the homotopy theory on $\SSet$ we have worked out in \hyperref[ch:simplicial_homotopy]{Chapter~\ref*{ch:simplicial_homotopy}}.

  In the Kan model structure $\SSet$, the fibrant objects are Kan complexes, and every object is cofibrant. Therefore, the fibrant-cofibrant objects are simply Kan complexes.

  The homotopy category $\mathrm{Ho}(\SSet)$ therefore simply has objects given by Kan complex, and morphisms given by equivalence classes of morphisms under homotopy. We will meet this category later. It is called the \emph{homotopy category of spaces.}
\end{example}

\section{Fibrant-Cofibrant replacement}
\label{sec:fibrant_cofibrant_replacement}

Let $\mathcal{C}$ be a model category.

In \hyperref[ssc:cofibrant_replacement]{Subsection~\ref*{ssc:cofibrant_replacement}}, we give a way of assigning to any object $X \in \mathcal{C}$ a (not necessarily unique) cofibrant object $QX$ to which it is weakly equivalent. We call any choice of such at cofibrant object a \emph{cofibrant replacement.}
%and show that replacing objects by cofibrant replacements preserves left homotopy, in the sense that
%\begin{equation*}
%  X \overset{l}{\sim} Y \implies QX \overset{l}{\sim} QY.
%\end{equation*}

We also define the notion of cofibrant replacement on morphisms, i.e.\ a way of assigning
\begin{equation*}
  (f\colon X \to Y) \mapsto (Qf\colon QX \to QY).
\end{equation*}
Again, this assignment is not uniquely defined, so cofibrant does not yield a functor, as one might hope.

Next we next prove that for any cofibrant replacements $Qf$ of $f$ and $Qg$ of $g$,
\begin{equation*}
  f \overset{l}{\sim} g \implies Qf \overset{l}{\sim} Qg,
\end{equation*}
showing that the cofibrant replacement of a morphism is uniquely defined up to homotopy if we restrict our attention to morphisms out of cofibrant objects $X$. In this setting, we have the further nice property that cofibrant replacement respects composition in the sense that
\begin{equation*}
  Q(g \circ f) \overset{l}{\sim} Qg \circ Qf.
\end{equation*}

In \hyperref[ssc:fibrant_replacement]{Subsection~\ref*{ssc:fibrant_replacement}}, we study the dual notion, called \emph{fibrant replacement,} providing a method by which one can replace any object $X$ by a fibrant object $RX$ to which it is weakly equivalent. We then again define a notion of cofibrant replacement on morphims, i.e. a way of assigning
\begin{equation*}
  (f\colon X \to Y) \mapsto (Rf\colon RX \to RY).
\end{equation*}
Then, restricting to morphisms whose domain $Y$ is fibrant, and considering morphisms up to right homotopy, we find that fibrant replacement becomes well-defined in the sense that
\begin{equation*}
  f \overset{r}{\sim} f' \implies Rf \overset{r}{\sim} Rf',
\end{equation*}
and preserves composition in the sense that $R(g \circ f) = Rg \circ Rf$.

In \hyperref[ssc:fibrant_cofibrant_replacement]{Subsection~\ref*{ssc:fibrant_cofibrant_replacement}}, we prove two preliminary lemmas, which show the following.
\begin{itemize}
  \item The fibrant replacement of a cofibrant object is both fibrant and cofibrant.

  \item Fibrant replacement takes left-homotopic morphisms $f$, $g\colon X \to Y$ with $X$ cofibrant to morphisms which are right homotopic.
\end{itemize}
The first lemma implies that taking first the cofibrant and then fibrant replacement of any object $X$ yields an object $RQX$ which is both fibrant and cofibrant; one calls such objects \emph{fibrant-cofibrant,} and the act of replacing an object by a weakly equivalent fibrant-cofibrant one \emph{fibrant-cofibrant replacement.} The second lemma then implies that fibrant-cofibrant replacement gives a functor from $\mathcal{C}$ to $\mathrm{Ho}(\mathcal{C})$.

\subsection{Cofibrant replacement}
\label{ssc:cofibrant_replacement}

\begin{definition}[cofibrant replacement]
  \label{def:cofibrant_replacement}
  Let $\mathcal{C}$ be a model category. For each $X \in \mathcal{C}$, a \defn{cofibrant replacement} of $X$ consists of a cofibrant object $QX$ and a trivial fibration $p_{X}\colon QX \to X$, defined as follows.
  \begin{itemize}
    \item If $X$ is cofibrant, define $QX = X$, and $p_{X} = \id_{X}$.

    \item If $X$ is not cofibrant, pick a factorization of the unique morphism $\emptyset \to X$
      \begin{equation*}
        \emptyset \to QX \overset{p_{X}}{\to} X
      \end{equation*}
      where the map $\emptyset \to Qx$ is a cofibration and $p_{X}$ is a trivial fibration.
  \end{itemize}

  For a morphism $f\colon X \to Y$, a fibrant replacement of $f$ is any solution $Rf$ to the following lifting problem.
  \begin{equation*}
    \begin{tikzcd}
      \emptyset
      \arrow[r]
      \arrow[d]
      & QY
      \arrow[d, "p_{Y}"]
      \\
      QX
      \arrow[r, swap, "f \circ p_{X}"]
      \arrow[ur, dashed, "Qf"]
      & Y
    \end{tikzcd}
  \end{equation*}
\end{definition}

\begin{example}
  Consider the category $\mathbf{Ch}_{\geq}(R)$ of bounded-below chain complexes of modules over a ring $R$. Let $M$ be an $R$-module, considered as a chain complex concentrated in degree 0. A cofibrant replacement for $M$ is a chain complex of projective modules $P^{M}$ together with a trivial fibration $P^{M} \to M$. This is the same thing as a projective resolution of $M$.
\end{example}

\begin{lemma}
  \label{lemma:cofibrant_replacement_respects_left_homotopy}
  For any map $f\colon X \to Y$, any choice of cofibrant replacement $f \mapsto Qf$ respects left homotopy; that is, if $f \overset{l}{\sim} f'$, then $Qf \overset{l}{\sim} Qf'$.
\end{lemma}
\begin{proof}
  Suppose we have lifts $Qf$ and $Qf'$ of $f$ and $f'$ respectively, and consider a left homotopy between them.
  \begin{equation*}
    \begin{tikzcd}
      X \amalg X
      \arrow[d, swap, "i"]
      \arrow[r, "{(f, f')}"]
      & Y
      \\
      C
      \arrow[ur, swap, "H"]
    \end{tikzcd}
  \end{equation*}

  Consider the following solid lifting problem.
  \begin{equation*}
    \begin{tikzcd}
      QX \amalg QX
      \arrow[r, "{(Qf, Qf')}"]
      \arrow[d, swap, "i"]
      & QY
      \arrow[d, "p_{Y}"]
      \\
      C
      \arrow[r, swap, "f \circ p_{X} \circ H"]
      \arrow[ur, dashed]
      & Y
    \end{tikzcd}
  \end{equation*}
  The outer square commutes because
  \begin{align*}
    p_{Y} \circ (Qf, Qf') &= (f \circ p_{X}, f' \circ p_{X})
    \\
    &= f \circ p_{X} \circ (f, f')
    \\
    &= f \circ p_{X} \circ H \circ i.
  \end{align*}

  The map $i$ is a cofibration by definition, and $p_{Y}$ is a trivial fibration. Thus, we get a dashed lift. But this is precisely a left homotopy between $Qf$ and $Qf'$.
\end{proof}

\begin{corollary}
  \label{cor:cofibrant_replacement_respects_composition}
  We have
  \begin{equation*}
    Q(g \circ f) \overset{l}{\sim} Q(g) \circ Q(f).
  \end{equation*}
\end{corollary}
\begin{proof}
  Consider the following square.
  \begin{equation*}
    \begin{tikzcd}
      *
      \arrow[r]
      \arrow[d]
      & *
      \arrow[r]
      \arrow[d]
      & QZ
      \arrow[d, "p_{Z}"]
      \\
      *
      \arrow[r]
      \arrow[d]
      & QY
      \arrow[r, "g \circ p_{Y}"]
      \arrow[d, "p_{Y}"]
      & Z
      \arrow[d, "\id"]
      \\
      QX
      \arrow[r, swap, "f \circ p_{X}"]
      & Y
      \arrow[r, swap, "g"]
      & Z
    \end{tikzcd}
  \end{equation*}
  The outer square is the lifting problem defining $Q(g \circ f)$, and the top right and bottom left squares define $Qg$ and $Qf$ respectively. Thus, $Qg \circ Qf \overset{l}{\sim} Q(g \circ f)$.
\end{proof}

There is a dual notion.

\subsection{Fibrant replacement}
\label{ssc:fibrant_replacement}

\begin{definition}[fibrant replacement]
  \label{def:fibrant_replacement}
  Let $\mathcal{C}$ be a model category. For each $X \in \mathcal{C}$, a \defn{fibrant replacement} of $X$ consists of a fibrant object $RX$ and a trivial cofibration $i_{X}\colon X \to RX$, defined as follows.
  \begin{itemize}
    \item If $X$ is fibrant, define $RX = X$, and $i_{X} = \id_{X}$.

    \item If $X$ is not fibrant, pick factorization of the unique morphism $X \to *$
      \begin{equation*}
        X \overset{i_{X}}{\to} RX \to *
      \end{equation*}
      where the map $\emptyset \to Qx$ is a fibration and $i_{X}$ is a trivial cofibration.
  \end{itemize}

  For a morphism $f\colon X \to Y$, a fibrant replacement of $f$ is a solution $Rf$ to the following lifting problem.
  \begin{equation*}
    \begin{tikzcd}
      X
      \arrow[r, "i_{Y} \circ f"]
      \arrow[d, swap, "i_{X}"]
      & RY
      \arrow[d]
      \\
      RX
      \arrow[r]
      \arrow[ur, dashed, "Rf"]
      & *
    \end{tikzcd}
  \end{equation*}
\end{definition}

\begin{lemma}
  \label{lemma:fibrant_replacement_respects_right_homotopy}
  For any map $f\colon X \to Y$, fibrant replacement $f \mapsto Rf$ respects right homotopy; that is, if $f \overset{r}{\sim} f'$, then $Rf \overset{r}{\sim} Rf'$.
\end{lemma}
\begin{proof}
  Dual to \hyperref[lemma:cofibrant_replacement_respects_left_homotopy]{Lemma~\ref*{lemma:cofibrant_replacement_respects_left_homotopy}}.
\end{proof}

\begin{corollary}
  We have
  \begin{equation*}
    R(g \circ f) \overset{r}{\sim} R(g) \circ R(f).
  \end{equation*}
\end{corollary}
\begin{proof}
  Dual to \hyperref[cor:cofibrant_replacement_respects_composition]{Corollary~\ref*{cor:cofibrant_replacement_respects_composition}}
\end{proof}

\subsection{Fibrant-cofibrant replacement}
\label{ssc:fibrant_cofibrant_replacement}

\begin{lemma}
  \label{lemma:fibrant_replacement_turns_left_homotopy_from_cofibrant_into_right_homotopy}
  Suppose $f$, $f'\colon X \to Y$ are left homotopic, and that $X$ is cofibrant. Then the fibrant replacements $Rf$ and $Rf'$ are right homotopic.
\end{lemma}
\begin{proof}
  Since $f$ and $f'$ are left homotopic, so are $i_{Y} \circ f$ and $i_{Y} \circ f'$. But since these are between a fibrant and a cofibrant object, they are right homotopic. Pick some right homotopy $H$ as follows.
  \begin{equation*}
    \begin{tikzcd}[row sep=huge, column sep=huge]
      & P
      \arrow[d, "p"]
      \\
      X
      \arrow[ur, "H"]
      \arrow[r, swap, "{(i_{Y} \circ f, i_{Y} \circ f')}"]
      & RY \times RY
    \end{tikzcd}
  \end{equation*}

  Consider the following solid lifting problem.
  \begin{equation*}
    \begin{tikzcd}
      X
      \arrow[r, "H"]
      \arrow[d, swap, "i_{X}"]
      & R
      \arrow[d, "p"]
      \\
      RX
      \arrow[r, swap, "{(Rf, Rf')}"]
      \arrow[ur, dashed]
      & RY \times RY
    \end{tikzcd}
  \end{equation*}
  The outer square commutes because
  \begin{equation*}
    p \circ H = (i_{Y} \circ f, i_{Y} \circ f') = (Rf \circ i_{X}, Rf' \circ i_{X}).
  \end{equation*}
\end{proof}

\begin{theorem}
  There is a functor $\pi\colon \mathcal{C} \to \mathrm{Ho}(\mathcal{C})$, defined on objects by $X \mapsto RQX$ and on morphisms by $f \mapsto [RQf]$.
\end{theorem}
\begin{proof}
  First, note that the assignment on objects is well-defined since $RQX$ is fibrant by definition and cofibrant because the composition
  \begin{equation*}
    \emptyset \to QX \overset{i_{X}}{\to} RQX
  \end{equation*}
  is a cofibration.

  It remains to check that $\pi$ respects composition. To see this, note that
  \begin{align*}
    \pi(g \circ f) &= [RQ(g \circ f)] \\
    &= [R(Qg \circ Qf)] \\
    &= [RQ(g) \circ RQ(f)] \\
    &= [RQ(g)] \circ [RQ(f)] \\
    &= \pi(g) \circ \pi(f),
  \end{align*}
  where the second equality follows from \hyperref[cor:cofibrant_replacement_respects_composition]{Corollary~\ref*{cor:cofibrant_replacement_respects_composition}} and \hyperref[lemma:fibrant_replacement_turns_left_homotopy_from_cofibrant_into_right_homotopy]{Lemma~\ref*{lemma:fibrant_replacement_turns_left_homotopy_from_cofibrant_into_right_homotopy}} and the third follows from \hyperref[lemma:fibrant_replacement_respects_right_homotopy]{Lemma~\ref*{lemma:fibrant_replacement_respects_right_homotopy}}.
\end{proof}

\begin{theorem}
  The homotopy category $\mathrm{Ho}(\mathcal{C})$, together with the map $\pi\colon \mathcal{C} \to \mathrm{Ho}(\mathcal{C})$, is the localization of $\mathcal{C}$ at $\mathcal{W} = \W$, the set of weak equivalences. That is, there is an equivalence of categories $\mathcal{C}[\mathcal{W}^{-1}] \simeq \mathrm{Ho}(\mathcal{C})$.
\end{theorem}
\begin{proof}
  Define a category $\overline{\mathcal{C}}$ as follows.
  \begin{itemize}
    \item $\Obj(\overline{\mathcal{C}}) = \Obj(\mathcal{C})$

    \item For each $X$, $Y \in \Obj(\overline{\mathcal{C}}), \overline{\mathcal{C}}(X, Y) = \mathrm{Ho}(\mathcal{C})(RQX, RQY)$.
  \end{itemize}

  We have the following commutative diagram
  \begin{equation*}
    \begin{tikzcd}
      \mathcal{C}
      \arrow[rr, "\pi"]
      \arrow[dr, swap, "\bar{\pi}"]
      && \mathrm{Ho}(\mathcal{C})
      \\
      & \overline{\mathcal{C}}
      \arrow[ur, swap, "\xi"]
    \end{tikzcd}
  \end{equation*}
  where $\bar{\pi}$ is the identity on objects and maps $f \mapsto [RQf]$, and $\xi$ acts on objects by taking $X \mapsto RQX$ and is the identity on morphisms.

  Note that $\xi$ is surjective on objects and fully faithful, hence an equivalence of categories. Hence, we only need to show that $\overline{\mathcal{C}}$ satisfies the universal property for the localization.

  Suppose that $F\colon \mathcal{C} \to \mathcal{D}$ is a functor which takes weak equivalences to isomorphisms in $\mathcal{D}$. If we can show that there exists a unique functor $\overline{F}\colon \overline{\mathcal{C}} \to \mathcal{D}$ such that $F = \overline{F} \circ \overline{\pi}$, then we are done.

  Suppose $f \overset{r}{\sim} g$. Pick some path object $P$, and a homotopy through $P$.
  \begin{equation*}
    \begin{tikzcd}
      & P
      \arrow[d, "p"]
      \\
      X
      \arrow[r, swap, "{(\id, \id)}"]
      \arrow[ur, "s"]
      & X \times X
    \end{tikzcd}
    \qquad
    \begin{tikzcd}
      & P
      \arrow[d, "p"]
      \\
      X
      \arrow[r, swap, "{(f, g)}"]
      \arrow[ur, "H"]
      & Y \times Y
    \end{tikzcd}
  \end{equation*}
  Putting these diagrams back to back and unfolding them, we have the following.
  \begin{equation*}
    \begin{tikzcd}
      & Y
      \\
      X
      \arrow[ur, "f"]
      \arrow[r, "H"]
      \arrow[dr, swap, "g"]
      & P
      \arrow[u, "p_{0}"]
      \arrow[d, swap, "p_{1}"]
      & Y
      \arrow[ul, swap, "\id"]
      \arrow[l, swap, "s"]
      \arrow[dl, "\id"]
      \\
      & Y
    \end{tikzcd}
  \end{equation*}

  Consider the image in $\mathcal{D}$. Since $p_{0}$ and $p_{1}$ are weak equivalences, \hyperref[thm:whitehead]{Theorem~\ref*{thm:whitehead}} implies that they are sent to isomorphisms in $\mathcal{D}$. Thus, $F(p_{0}) = F(s)^{-1} = F(p_{1})$, so
  \begin{align*}
    F(f) &= F(p_{0}) \circ F(H) \\
    &= F(p_{1}) \circ F(H) \\
    &= F(g).
  \end{align*}

  Let $[g]\colon RQX \to RQY$ be a morphism in $\overline{\mathcal{C}}$. Pick a representative $g$. Consider the following commutative diagram.
  \begin{equation*}
    \begin{tikzcd}
      X
      \arrow[r, "g"]
      \arrow[d, leftarrow, swap, "p_{X}"]
      & Y
      \arrow[d, leftarrow, "p_{Y}"]
      \\
      QX
      \arrow[r, "Qg"]
      \arrow[d, swap, "i_{QX}"]
      & QY
      \arrow[d, "i_{QY}"]
      \\
      RQX
      \arrow[r, "RQg"]
      & RQY
    \end{tikzcd}
  \end{equation*}

  Applying $F$, we find the diagram
  \begin{equation*}
    \begin{tikzcd}
      F(X)
      \arrow[r, "F(g)"]
      \arrow[d, leftarrow, swap, "F(p_{X})"]
      & F(Y)
      \arrow[d, leftarrow, "F(p_{Y})"]
      \\
      F(QX)
      \arrow[r, "F(Qg)"]
      \arrow[d, swap, "F(i_{QX})"]
      & F(QY)
      \arrow[d, "F(i_{QY})"]
      \\
      F(RQX)
      \arrow[r, "F(RQg)"]
      & F(RQY)
    \end{tikzcd}
  \end{equation*}
  But the vertical morphisms are isomorphisms, so
  \begin{equation*}
    F(RQg) = F(i_{QY}) \circ F(p_{Y})^{-1} \circ F(g) \circ F(p_{X}) \circ F(i_{QX})^{-1}.
  \end{equation*}

  In order to have $F = \bar{F} \circ \bar{\pi}$, we must have
  \begin{equation*}
    \overline{F}([RQg]) = F(RQg),
  \end{equation*}
  which specifies $\bar{F}$ completely.
\end{proof}

\section{Quillen adjunctions}
\label{sec:quillen_adjunctions}

\begin{proposition}
  \label{prop:equivalent_conditions_for_quillen_adjunction}
  Let $\mathcal{C}$ and $\mathcal{D}$ be model categories, and consider an adjunction
  \begin{equation*}
    F : \mathcal{C} \leftrightarrow \mathcal{D} : G.
  \end{equation*}
  The following are equivalent.
  \begin{enumerate}
    \item $F$ preserves cofibrations and trivial cofibrations.

    \item $G$ preserves fibrations and trivial fibrations.
  \end{enumerate}
\end{proposition}
\begin{proof}
  Suppose 2.\ holds. By \hyperref[lemma:relations_between_lifting_properties_and_fibrations]{Lemma~\ref*{lemma:relations_between_lifting_properties_and_fibrations}}, a morphism is a cofibration if and only if it has the left lifting property with respect to all trivial fibrations. Thus, in order to show that $F$ preserves cofibrations, it suffices to show that if $f$ is a cofibration in $\mathcal{C}$, then $F(f)$ has the left lifting property with respect to all trivial fibrations.

  Consider solid adjunct diagrams of the following form, with the first in $\mathcal{D}$ and the second in $\mathcal{C}$.
  \begin{equation*}
    \begin{tikzcd}
      F(c)
      \arrow[r]
      \arrow[d, swap, "F(f)"]
      & d
      \arrow[d, "g"]
      \\
      F(c')
      \arrow[r]
      \arrow[ur, dashed]
      & d'
    \end{tikzcd}
    \qquad
    \begin{tikzcd}
      c
      \arrow[r]
      \arrow[d, swap, "f"]
      & G(d)
      \arrow[d, "G(g)"]
      \\
      c'
      \arrow[r]
      \arrow[ur, dashed]
      & G(d')
    \end{tikzcd}
  \end{equation*}
  Suppose that $f$ is a cofibration and that $F$ preserves cofibrations. Then the left-hand diagram has a lift for any trivial fibration $g$. The adjunct of this lift is a lift for the diagram on the right. This means that $G(g) \in \cof\rlp = \fib \cap \W$.

  Similarly, if $f$ is a trivial cofibration and $F$ preserves trivial cofibrations, then the left-hand diagram has a lift for any fibration $g$, whose adjunct is a lift for the diagram on the RHS, implying that $G(d) \in (\cof \cap \W)\rlp = \fib$.

  The other statement is dual.
\end{proof}

\begin{definition}[Quillen adjunction]
  \label{def:quillen_adjunction}
  An adjunction of model categories satisfying the equivalent conditions in \hyperref[prop:equivalent_conditions_for_quillen_adjunction]{Proposition~\ref*{prop:equivalent_conditions_for_quillen_adjunction}} is known as a \defn{Quillen adjunction}.
\end{definition}

\begin{lemma}[Ken Brown's Lemma]
  \label{lemma:ken_browns_lemma}
  Let $F : \mathcal{C} \leftrightarrow \mathcal{D} : G$ be a Quillen adjunction.
  \begin{enumerate}
    \item The left adjoint $F$ preserves weak equivalences between cofibrant objects

    \item The right adjoint $G$ preserves weak equivalences between fibrant objects.
  \end{enumerate}
\end{lemma}
\begin{proof}
  By \hyperref[prop:equivalent_conditions_for_quillen_adjunction]{Proposition~\ref*{prop:equivalent_conditions_for_quillen_adjunction}}. Let $A$ and $B$ be cofibrant objects, and $f\colon A \to B$ be a weak equivalence.

  Pick for the map $(f, \id)\colon A \amalg B \to B$ a factorization
  \begin{equation*}
    A \amalg B \overset{i}{\to} C \overset{p}{\to} B,
  \end{equation*}
  where $i \in \cof$ and $p \in \fib \cap \W$.

  Consider the following diagram.
  \begin{equation*}
    \begin{tikzcd}
      \emptyset
      \arrow[r]
      \arrow[d]
      & A
      \arrow[dr, bend left, "f"]
      \arrow[d, "i_{A}"]
      \\
      B
      \arrow[r, swap, "i_{B}"]
      & A \amalg B
      \arrow[r, "{(f, \id)}"]
      \arrow[d, swap, "q"]
      & B
      \\
      & C
      \arrow[ur, swap, "p"]
    \end{tikzcd}
  \end{equation*}

  Because $A$ and $B$ are cofibrant, $i_{A}$ and $i_{B}$ are cofibrations. Thus, $q \circ i_{A}$ is a cofibration, so $C$ is cofibrant.

  Note that by commutativity, we have
  \begin{equation*}
    f = p \circ q \circ i_{A},\qquad \id_{B} = p \circ q \circ i_{B}.
  \end{equation*}
  By \hyperref[item:MC_two_out_of_three]{Axiom~\ref*{item:MC_two_out_of_three}}, we have that $q \circ i_{A}$ and $q \circ i_{B}$ are weak equivalences, hence trivial cofibrations.

  Applying $F$, we find that
  \begin{equation*}
    F(f) = F(p) \circ F(q \circ i_{A}),\qquad F(\id) = F(p) \circ F(q \circ i_{B}).
  \end{equation*}
  We know that $F(\id_{B}) = \id_{F(b)}$ is a trivial cofibration, hence certainly a weak equivalence, and that $F(q \circ i_{B})$ and $F(q \circ i_{A})$ are a weak equivalence because $F$ preserves trivial cofibrations. Thus $F(p)$ is a weak equivalence, implying that
  \begin{equation*}
    F(f) = F(p) \circ F(q \circ i_{A})
  \end{equation*}
  is a weak equivalence.
\end{proof}


Given a Quillen adjunction
\begin{equation*}
  F : \mathcal{C} \leftrightarrow \mathcal{D} : G,
\end{equation*}
consider the full subcategory of all cofibrant objects $\mathcal{C}_{c} \subset \mathcal{C}$. Since the full subcategory of $\mathcal{C}$ on fibrant-cofibrant objects is the same as the full subcategory of $\mathcal{C}_{c}$ on fibrant-cofibrant objects, there is an equivalence of categories
\begin{equation*}
  \mathcal{C}_{c}[\mathcal{W}^{-1}] \simeq \mathcal{C}[\mathcal{W}^{-1}].
\end{equation*}

We will denote any weak inverse by
\begin{equation*}
  Q\colon \mathcal{C}[\mathcal{W}^{-1}] \to \mathcal{C}_{c}[\mathcal{W}^{-1}].
\end{equation*}

Dually, we get a functor
\begin{equation*}
  R\colon \mathcal{D}[\mathcal{W}^{-1}] \to \mathcal{D}_{f}[\mathcal{W}^{-1}],
\end{equation*}
where $\mathcal{D}_{f}$ is the full subcategory of $\mathcal{D}$ on fibrant objects.

\begin{definition}[derived functor]
  \label{def:left_derived_functor}
  Let
  \begin{equation*}
    F : \mathcal{C} \leftrightarrow \mathcal{D} : G
  \end{equation*}
  be a Quillen adjunction.
  \begin{itemize}
    \item The \defn{left derived functor} $LF$ is the composite
      \begin{equation*}
        \begin{tikzcd}
          \mathcal{C}[\mathcal{W}^{-1}]
          \arrow[r, "Q"]
          & \mathcal{C}_{c}[\mathcal{W}^{-1}]
          \arrow[r, "F"]
          & \mathcal{D}[\mathcal{W}^{-1}].
        \end{tikzcd}
      \end{equation*}

    \item The \defn{right derived functor} $RG$ is the composite
      \begin{equation*}
        \begin{tikzcd}
          \mathcal{D}[\mathcal{W}^{-1}]
          \arrow[r, "R"]
          & \mathcal{D}_{f}[\mathcal{W}^{-1}]
          \arrow[r, "G"]
          & \mathcal{C}[\mathcal{W}^{-1}].
        \end{tikzcd}
      \end{equation*}
  \end{itemize}
\end{definition}

\begin{proposition}
  \label{prop:derived_functors_form_adjunction}
  Let
  \begin{equation*}
    \mathcal{C} : F \leftrightarrow \mathcal{G} : \mathcal{D}
  \end{equation*}
  be a Quillen adjunction. Then there is an adjunction of derived functors
  \begin{equation*}
    LF : \mathcal{C}[\mathcal{W}^{-1}] \leftrightarrow \mathcal{D}[\mathcal{W}^{-1}] : RG.
  \end{equation*}
\end{proposition}

\begin{definition}[Quillen equivalence]
  \label{def:quillen_equivalence}
  A Quillen adjunction is called a \defn{Quillen equivalence} if $LF$, or equivalently $RG$, is an equivalence of localized categories.
\end{definition}

\begin{theorem}
  The adjunction
  \begin{equation*}
    \abs{\cdot} : \SSet \leftrightarrow \mathbf{Top} : \Sing
  \end{equation*}
  is a Quillen adjunction.
\end{theorem}

\end{document}
