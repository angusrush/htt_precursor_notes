\documentclass[main.tex]{subfiles}

\begin{document}

\chapter{Introduction}
\label{ch:introduction}

These notes grew out of a course given by Prof.\ Tobias Dyckerhoff at the University of Hamburg. However, much of the presentation and many of the proofs are heavily adapted. All mistakes are entirely my own.

They are under very heavy construction at the moment.

\section{Axiomatic framwork}
\label{sec:axiomatic_framwork}

These notes are phrased in terms of $\mathbf{ZFCU}$, i.e.\ the Zermelo-Frenkel axioms together with the following axioms.
\begin{itemize}
  \item $\mathbf{C}$: the axiom of choice.

  \item $\mathbf{U}$: Grothendieck's universe axiom.
\end{itemize}

\section{Notation}
\label{sec:notation}

\begin{itemize}
  \item The unit and counit are sometimes called $\epsilon$ and $\eta$ respectively, and sometimes the other way around. I'm going through and fixing this.

  \item We denote hom-sets $\mathcal{C}(a, b)$.
\end{itemize}

\end{document}
